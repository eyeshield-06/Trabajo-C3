\documentclass{article}
\usepackage[spanish]{babel}
\usepackage{amsmath}
\usepackage{fancybox}
\usepackage{graphicx}
\usepackage{hyperref}
%%%%%%
% no incluyan mas paquetes hasta que los vayan a ocupar.
% de lo contrario compila mas lento.
%%%%%%
\begin{document}
    %%%%%%%%%%
    %% Portada
    \begin{figure}
        \includegraphics[scale=0.25]{/home/eyeshield6/Documents/compu3/trabajo/LaTeX/images/LOGO.png}
    \end{figure}
    \begin{center}
        \shadowbox{\huge \textbf{Trabajo Investigacion:}}\\[0.2cm]
        \shadowbox{\huge \textbf{Computacion Estadistica III}}\\[1cm]
        \begin{minipage}{0.4\textwidth}
            \begin{flushleft} \large
                \emph{\textbf{Docente:}}\\
                \textup{Luis Guzman}
            \end{flushleft}
        \end{minipage}
        ~
        \begin{minipage}{0.4\textwidth}
            \begin{flushright} \large
                \emph{\textbf{Estudiantes:}} \\
                \textup{Felipe \'Avila}\\
                \textup{Francisca Pacheco}\\
                \textup{Paula Quilodran}\\
                \textup{Rudy Miranda}
            \end{flushright}
        \end{minipage}\\[1cm]
        \makeatother
    \end{center}
    %% Portada
    %%%%%%%%%%
    
    \tableofcontents

    \listoffigures

    \listoftables

    \section{Introduccion}
    
    Como en R existen funciones que requieren mucho tiempo de ejecución como por ejemplo la clusterización se busca una alternativa para reducir este tiempo, de esta manera nacen funciones como .C, .Call y .External dado que R contiene lenguaje C es posible utilizar dichas funciones.
    Se busca conocer las funciones anterior mente mencionadas debido a sus diferencias en estructura y complejidad de uso. 

    \section{Funciones \textit{.C}, \textit{.Call} y \textit{.External}}

    \subsection{\textit{.C}}
    
    Es una función que conecta C/C++ con R condicionada a los argumentos y parámetros que le entreguemos a la función desde R \textbf{NO} serán reconocidos inmediatamente por el lenguaje C, por ende debemos especificar y transformar los argumentos con funciones como \textbf{as.integer()}, \textbf{as.logical()}, entre otras. Esta función no retorna valores.
    
    \subsection{\textit{.Call}}
    
    Es una función que conecta C/C++ con R condicionada a que los argumentos y parámetros que le entreguemos a la función desde R \textbf{SI} serán reconocidos inmediatamente por el lenguaje C, permitiendo entregar argumentos sin transformación y recibe todo tipo de argumentos, permite retornar valores.
    
    \subsection{\textit{.External}}

    Es una función que conecta C/C++ con R condicionada a que los argumentos y parámetros que le entreguemos a la función desde R \textbf{SI} serán reconocidos inmediatamente por el lenguaje C,recibe solo un tipo de argumento que es una  \textit{pairlist}, permite retornar valores.

    \section{Funcion: Suma de Cuadrados}

    \subsection{Algoritmo}

    Implementacion de formula de Calculo,

    \begin{equation*}
        \sum_{i = 1}^{n} x_{i}^{2}
    \end{equation*}
    
    Donde $n$ corresponde a la longitud del vector.

    \subsection{\textit{R}}

    \href{https://github.com/eyeshield-06/Trabajo-C3/blob/master/source/exercise2/a/function.R}{source}

    \subsection{\textit{C}}

    \href{https://github.com/eyeshield-06/Trabajo-C3/tree/master/source/exercise2/b}{folder}

    \subsection{Test}
    
    \section{Optimizacion funcion \textit{dist}}

    \subsection{Algoritmo}

    Dados dos vectores de igual longitud, la componente $ij$ de la matrix de distacia  $D$, es

    \begin{equation*}
        D_{ij} = \sqrt{ \sum_{k = 1}^{n} (X_{ik} - Y_{jk})^2 }
    \end{equation*}

    Donde $n$ es la dimension del espacio vectorial.
    
    \subsection{\textit{R}}
    
    \subsection{\textit{C}}

    \subsection{Test}
    
    \section{Conclusion}

    \section{Bibliografia}
\end{document}
